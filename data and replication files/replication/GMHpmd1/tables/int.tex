\begin{table}[H]
\scalebox{0.95}{
  \centerline{\begin{threeparttable}
  \caption{Summary of the ``Parents Make the Difference'' curriculum}
  \label{tbl:int}
  \centering
  \begin{tabular}{p{0.5in}p{5in}}
  \toprule
  Session & Topic \\
  \midrule
  1 & \textbf{\textit{Introduction to nurturing and positive parenting}} \\
    & Welcome caregivers and provide an overview of the program. Explore the caregivers' own childhoods and their experience of parenting. Discuss their goals for their children. \\
  2 & \textbf{\textit{Childhood development and appropriate expectations}} \\
    & Provide psychoeducation regarding child development and age-appropriate expectations. Review the influence of environmental factors on child cognitive, emotional, and social development. Introduce the concept of praise and how it can promote positive child functioning. \\
  3 & \textbf{\textit{Communication with children and empathetic listening}} \\
    & Discuss effective communication strategies with young children and the use of play to teach and communicate with children. Introduce the concept of empathy and the importance of mutual respect between parents and children. \\
  4 & \textbf{\textit{Discipline with dignity}} \\
    & Discuss the importance of positive discipline. Review and practice positive behavior management skills, including praise, ignoring, and timeout. \\
  5 & \textbf{\textit{Activities to promote academic readiness}} \\
    & Review and practice simple activities, such as story-telling and word games, that can promote a child's cognitive and academic development. Encourage parental involvement in their child's school activities. academics. \\
  6 & \textbf{\textit{Malaria prevention}} \\
    & Review the causes and dangers of malaria and why children are especially vulnerable. Discuss prevention methods and appropriate response to early symptoms. \\
  7 & \textbf{\textit{Academic games: Making learning fun!}} \\
    & Build on Session 5 and review and practice more specific academic games, emphasizing early math skills and fine motor skills. \\
  8 & \textbf{\textit{Establishing routines and house rules}} \\
    & Discuss the importance of positive routines and rules for young children. \\
  9 & \textbf{\textit{Parent self-care and stress management}} \\
    & Review the concept of parent self-care and stress management. Introduce basic relaxation exercises and the concept of positive thinking. \\
  10 & \textbf{\textit{Wrap Up: Summarize lessons learned and celebrate successes!}} \\
    & Summarize the highlights of the program and praise caregivers for their positive progress. \\ 
  \bottomrule
  \end{tabular}
  \begin{tablenotes}
  \small
  \item Note. Each session lasted approximately two hours. Sessions were designed to be highly interactive with a strong emphasis on discussion, modeling, and in-session practice of skills.
  \end{tablenotes}
  \end{threeparttable}}}
\end{table}
